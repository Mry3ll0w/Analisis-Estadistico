% Options for packages loaded elsewhere
\PassOptionsToPackage{unicode}{hyperref}
\PassOptionsToPackage{hyphens}{url}
\documentclass[
]{article}
\usepackage{xcolor}
\usepackage[margin=1in]{geometry}
\usepackage{amsmath,amssymb}
\setcounter{secnumdepth}{-\maxdimen} % remove section numbering
\usepackage{iftex}
\ifPDFTeX
  \usepackage[T1]{fontenc}
  \usepackage[utf8]{inputenc}
  \usepackage{textcomp} % provide euro and other symbols
\else % if luatex or xetex
  \usepackage{unicode-math} % this also loads fontspec
  \defaultfontfeatures{Scale=MatchLowercase}
  \defaultfontfeatures[\rmfamily]{Ligatures=TeX,Scale=1}
\fi
\usepackage{lmodern}
\ifPDFTeX\else
  % xetex/luatex font selection
\fi
% Use upquote if available, for straight quotes in verbatim environments
\IfFileExists{upquote.sty}{\usepackage{upquote}}{}
\IfFileExists{microtype.sty}{% use microtype if available
  \usepackage[]{microtype}
  \UseMicrotypeSet[protrusion]{basicmath} % disable protrusion for tt fonts
}{}
\makeatletter
\@ifundefined{KOMAClassName}{% if non-KOMA class
  \IfFileExists{parskip.sty}{%
    \usepackage{parskip}
  }{% else
    \setlength{\parindent}{0pt}
    \setlength{\parskip}{6pt plus 2pt minus 1pt}}
}{% if KOMA class
  \KOMAoptions{parskip=half}}
\makeatother
\usepackage{color}
\usepackage{fancyvrb}
\newcommand{\VerbBar}{|}
\newcommand{\VERB}{\Verb[commandchars=\\\{\}]}
\DefineVerbatimEnvironment{Highlighting}{Verbatim}{commandchars=\\\{\}}
% Add ',fontsize=\small' for more characters per line
\usepackage{framed}
\definecolor{shadecolor}{RGB}{248,248,248}
\newenvironment{Shaded}{\begin{snugshade}}{\end{snugshade}}
\newcommand{\AlertTok}[1]{\textcolor[rgb]{0.94,0.16,0.16}{#1}}
\newcommand{\AnnotationTok}[1]{\textcolor[rgb]{0.56,0.35,0.01}{\textbf{\textit{#1}}}}
\newcommand{\AttributeTok}[1]{\textcolor[rgb]{0.13,0.29,0.53}{#1}}
\newcommand{\BaseNTok}[1]{\textcolor[rgb]{0.00,0.00,0.81}{#1}}
\newcommand{\BuiltInTok}[1]{#1}
\newcommand{\CharTok}[1]{\textcolor[rgb]{0.31,0.60,0.02}{#1}}
\newcommand{\CommentTok}[1]{\textcolor[rgb]{0.56,0.35,0.01}{\textit{#1}}}
\newcommand{\CommentVarTok}[1]{\textcolor[rgb]{0.56,0.35,0.01}{\textbf{\textit{#1}}}}
\newcommand{\ConstantTok}[1]{\textcolor[rgb]{0.56,0.35,0.01}{#1}}
\newcommand{\ControlFlowTok}[1]{\textcolor[rgb]{0.13,0.29,0.53}{\textbf{#1}}}
\newcommand{\DataTypeTok}[1]{\textcolor[rgb]{0.13,0.29,0.53}{#1}}
\newcommand{\DecValTok}[1]{\textcolor[rgb]{0.00,0.00,0.81}{#1}}
\newcommand{\DocumentationTok}[1]{\textcolor[rgb]{0.56,0.35,0.01}{\textbf{\textit{#1}}}}
\newcommand{\ErrorTok}[1]{\textcolor[rgb]{0.64,0.00,0.00}{\textbf{#1}}}
\newcommand{\ExtensionTok}[1]{#1}
\newcommand{\FloatTok}[1]{\textcolor[rgb]{0.00,0.00,0.81}{#1}}
\newcommand{\FunctionTok}[1]{\textcolor[rgb]{0.13,0.29,0.53}{\textbf{#1}}}
\newcommand{\ImportTok}[1]{#1}
\newcommand{\InformationTok}[1]{\textcolor[rgb]{0.56,0.35,0.01}{\textbf{\textit{#1}}}}
\newcommand{\KeywordTok}[1]{\textcolor[rgb]{0.13,0.29,0.53}{\textbf{#1}}}
\newcommand{\NormalTok}[1]{#1}
\newcommand{\OperatorTok}[1]{\textcolor[rgb]{0.81,0.36,0.00}{\textbf{#1}}}
\newcommand{\OtherTok}[1]{\textcolor[rgb]{0.56,0.35,0.01}{#1}}
\newcommand{\PreprocessorTok}[1]{\textcolor[rgb]{0.56,0.35,0.01}{\textit{#1}}}
\newcommand{\RegionMarkerTok}[1]{#1}
\newcommand{\SpecialCharTok}[1]{\textcolor[rgb]{0.81,0.36,0.00}{\textbf{#1}}}
\newcommand{\SpecialStringTok}[1]{\textcolor[rgb]{0.31,0.60,0.02}{#1}}
\newcommand{\StringTok}[1]{\textcolor[rgb]{0.31,0.60,0.02}{#1}}
\newcommand{\VariableTok}[1]{\textcolor[rgb]{0.00,0.00,0.00}{#1}}
\newcommand{\VerbatimStringTok}[1]{\textcolor[rgb]{0.31,0.60,0.02}{#1}}
\newcommand{\WarningTok}[1]{\textcolor[rgb]{0.56,0.35,0.01}{\textbf{\textit{#1}}}}
\usepackage{graphicx}
\makeatletter
\newsavebox\pandoc@box
\newcommand*\pandocbounded[1]{% scales image to fit in text height/width
  \sbox\pandoc@box{#1}%
  \Gscale@div\@tempa{\textheight}{\dimexpr\ht\pandoc@box+\dp\pandoc@box\relax}%
  \Gscale@div\@tempb{\linewidth}{\wd\pandoc@box}%
  \ifdim\@tempb\p@<\@tempa\p@\let\@tempa\@tempb\fi% select the smaller of both
  \ifdim\@tempa\p@<\p@\scalebox{\@tempa}{\usebox\pandoc@box}%
  \else\usebox{\pandoc@box}%
  \fi%
}
% Set default figure placement to htbp
\def\fps@figure{htbp}
\makeatother
\setlength{\emergencystretch}{3em} % prevent overfull lines
\providecommand{\tightlist}{%
  \setlength{\itemsep}{0pt}\setlength{\parskip}{0pt}}
\usepackage{bookmark}
\IfFileExists{xurl.sty}{\usepackage{xurl}}{} % add URL line breaks if available
\urlstyle{same}
\hypersetup{
  pdftitle={A1},
  pdfauthor={Antonio Roldán Andrade},
  hidelinks,
  pdfcreator={LaTeX via pandoc}}

\title{A1}
\author{Antonio Roldán Andrade}
\date{2025-10-27}

\begin{document}
\maketitle

\subsection{Disclaimer}\label{disclaimer}

Este es el documento de resolución de la actividad 1 para la asignatura
de Análisis Estadístico.

\section{1. Estructura de los datos}\label{estructura-de-los-datos}

\subsection{1.1 Diccionario de los
datos}\label{diccionario-de-los-datos}

En este apartado simplemente realizamos la carga de datos y una pequeña
ordenacion.

\begin{Shaded}
\begin{Highlighting}[]
\CommentTok{\# Cargamos en primer lugar el WorldSustainabilty}
\NormalTok{csv\_location }\OtherTok{\textless{}{-}} \StringTok{"WorldSustainabilityDataset.csv"}
\NormalTok{worldSustainabiltyDataSet }\OtherTok{\textless{}{-}} \FunctionTok{read.csv}\NormalTok{(csv\_location, }\AttributeTok{sep =} \StringTok{","}\NormalTok{)}

\CommentTok{\# Cargamos a continuación el DataDictonary}
\CommentTok{\# Para cargar el dataset usamos la libreria readxl ==\textgreater{} install.packages("readxl")}
\CommentTok{\# Realizamos la carga en la parte superior del código, limpieza}

\NormalTok{xlsx\_location }\OtherTok{\textless{}{-}} \StringTok{"Data\_Dictionary.xlsx"}
\NormalTok{data\_dictionaryDataSet }\OtherTok{\textless{}{-}} \FunctionTok{read\_excel}\NormalTok{(xlsx\_location)}


\CommentTok{\# Tabla 1 ==\textgreater{} ODS | Codigo Variable/Indicador | Descripcion}

\NormalTok{table1 }\OtherTok{\textless{}{-}}\NormalTok{ data\_dictionaryDataSet }\SpecialCharTok{\%\textgreater{}\%}
  \FunctionTok{select}\NormalTok{(}\StringTok{\textasciigrave{}}\AttributeTok{Associated SDG GOAL}\StringTok{\textasciigrave{}}\NormalTok{, Code, Description)}

\CommentTok{\# ODS | Descripcion}
\NormalTok{table2 }\OtherTok{\textless{}{-}}\NormalTok{ data\_dictionaryDataSet }\SpecialCharTok{\%\textgreater{}\%}
  \FunctionTok{select}\NormalTok{(}\StringTok{\textasciigrave{}}\AttributeTok{Associated SDG GOAL}\StringTok{\textasciigrave{}}\NormalTok{, Description)}

\CommentTok{\# Ordenamos y renombro por comodidad.}
\NormalTok{table1 }\OtherTok{\textless{}{-}}\NormalTok{ table1 }\SpecialCharTok{\%\textgreater{}\%} \FunctionTok{arrange}\NormalTok{(}\StringTok{\textasciigrave{}}\AttributeTok{Associated SDG GOAL}\StringTok{\textasciigrave{}}\NormalTok{)}
\NormalTok{table2 }\OtherTok{\textless{}{-}}\NormalTok{ table2 }\SpecialCharTok{\%\textgreater{}\%} \FunctionTok{arrange}\NormalTok{(}\StringTok{\textasciigrave{}}\AttributeTok{Associated SDG GOAL}\StringTok{\textasciigrave{}}\NormalTok{)}

\NormalTok{table1 }\OtherTok{\textless{}{-}}\NormalTok{ table1 }\SpecialCharTok{\%\textgreater{}\%} \FunctionTok{rename}\NormalTok{( }\AttributeTok{SDG =} \StringTok{\textasciigrave{}}\AttributeTok{Associated SDG GOAL}\StringTok{\textasciigrave{}}\NormalTok{)}
\NormalTok{table2 }\OtherTok{\textless{}{-}}\NormalTok{ table2 }\SpecialCharTok{\%\textgreater{}\%} \FunctionTok{rename}\NormalTok{( }\AttributeTok{SDG =} \StringTok{\textasciigrave{}}\AttributeTok{Associated SDG GOAL}\StringTok{\textasciigrave{}}\NormalTok{)}
\end{Highlighting}
\end{Shaded}

\begin{Shaded}
\begin{Highlighting}[]
\FunctionTok{head}\NormalTok{(table1,}\DecValTok{4}\NormalTok{)}
\end{Highlighting}
\end{Shaded}

\begin{verbatim}
## # A tibble: 4 x 3
##   SDG                  Code      Description                                    
##   <chr>                <chr>     <chr>                                          
## 1 Classification types Regime    Regime classified considering the competitiven~
## 2 Classification types Income    World Bank assigns the world’s economies to fo~
## 3 Classification types Region    World Region as classified by UN.              
## 4 Classification types Continent Continent classification according to UN Conve~
\end{verbatim}

\begin{Shaded}
\begin{Highlighting}[]
\FunctionTok{head}\NormalTok{(table2,}\DecValTok{4}\NormalTok{)}
\end{Highlighting}
\end{Shaded}

\begin{verbatim}
## # A tibble: 4 x 2
##   SDG                  Description                                              
##   <chr>                <chr>                                                    
## 1 Classification types Regime classified considering the competitiveness of acc~
## 2 Classification types World Bank assigns the world’s economies to four income ~
## 3 Classification types World Region as classified by UN.                        
## 4 Classification types Continent classification according to UN Convention
\end{verbatim}

\subsection{1.2 Fichero de datos}\label{fichero-de-datos}

Renombramos de forma manual Country y Country Code ya que no estan en el
diccionario

\begin{Shaded}
\begin{Highlighting}[]
\NormalTok{worldSustainabiltyDataSet }\OtherTok{\textless{}{-}}\NormalTok{ worldSustainabiltyDataSet }\SpecialCharTok{\%\textgreater{}\%}
  \FunctionTok{rename}\NormalTok{(}\AttributeTok{Country =} \StringTok{\textasciigrave{}}\AttributeTok{Country.Name}\StringTok{\textasciigrave{}}\NormalTok{)}

\NormalTok{worldSustainabiltyDataSet}\OtherTok{\textless{}{-}}\NormalTok{ worldSustainabiltyDataSet }\SpecialCharTok{\%\textgreater{}\%} 
  \FunctionTok{rename}\NormalTok{(}\AttributeTok{CountryCode =} \StringTok{\textasciigrave{}}\AttributeTok{Country.Code}\StringTok{\textasciigrave{}}\NormalTok{)}

\CommentTok{\# Recorremos los nombres y sustitumos basandonos en los códigos}
\CommentTok{\# de Dictionary}

\NormalTok{currColNames }\OtherTok{\textless{}{-}} \FunctionTok{names}\NormalTok{(worldSustainabiltyDataSet)}

\NormalTok{updateCurrNames }\OtherTok{\textless{}{-}} \ControlFlowTok{function}\NormalTok{(newColName, currColNames)\{}
\CommentTok{\# Primero obtenemos los códigos en base al campo field}
\NormalTok{regimeField\_Code }\OtherTok{\textless{}{-}}\NormalTok{ data\_dictionaryDataSet }\SpecialCharTok{\%\textgreater{}\%}
  \FunctionTok{filter}\NormalTok{(}\FunctionTok{str\_detect}\NormalTok{(Field,newColName)) }\SpecialCharTok{\%\textgreater{}\%}
  \FunctionTok{select}\NormalTok{(Code,Field)}

\CommentTok{\# Ahora sustituimos por la nueva }
\NormalTok{currColNames[}\FunctionTok{grep}\NormalTok{(regimeField\_Code}\SpecialCharTok{$}\NormalTok{Code, currColNames)]}\OtherTok{\textless{}{-}}\NormalTok{ newColName}
  \FunctionTok{return}\NormalTok{ (currColNames)}
\NormalTok{\}}

\CommentTok{\# Usamos la funcion }
\NormalTok{currColNames }\OtherTok{\textless{}{-}}  \FunctionTok{updateCurrNames}\NormalTok{(}\StringTok{\textquotesingle{}Regime\textquotesingle{}}\NormalTok{, currColNames)}

\NormalTok{currColNames }\OtherTok{\textless{}{-}} \FunctionTok{updateCurrNames}\NormalTok{(}\StringTok{\textquotesingle{}Income\textquotesingle{}}\NormalTok{,currColNames)}

\NormalTok{currColNames }\OtherTok{\textless{}{-}} \FunctionTok{updateCurrNames}\NormalTok{(}\StringTok{\textquotesingle{}Region\textquotesingle{}}\NormalTok{, currColNames)}

\FunctionTok{names}\NormalTok{(worldSustainabiltyDataSet) }\OtherTok{\textless{}{-}}\NormalTok{ currColNames}
\end{Highlighting}
\end{Shaded}

\begin{Shaded}
\begin{Highlighting}[]
\FunctionTok{head}\NormalTok{(}\FunctionTok{names}\NormalTok{(worldSustainabiltyDataSet),}\DecValTok{3}\NormalTok{)}
\end{Highlighting}
\end{Shaded}

\begin{verbatim}
## [1] "Country"     "CountryCode" "Year"
\end{verbatim}

\section{2 Tipos de datos y Posibles
Inconsistencias}\label{tipos-de-datos-y-posibles-inconsistencias}

\subsection{2.1 Variables Cuantitativas}\label{variables-cuantitativas}

Comenzaremos comprobando los valores centinelas y los valores nulos o
erroneos posibles.

\begin{Shaded}
\begin{Highlighting}[]
\NormalTok{cleanNames }\OtherTok{\textless{}{-}} \FunctionTok{names}\NormalTok{(worldSustainabiltyDataSet)}
\CommentTok{\# Comenzamos solo por los null}
\NormalTok{worldSustainabiltyDataSet }\OtherTok{\textless{}{-}}\NormalTok{ worldSustainabiltyDataSet }\SpecialCharTok{\%\textgreater{}\%} \FunctionTok{replace}\NormalTok{(.}\SpecialCharTok{==}\StringTok{"NULL"}\NormalTok{, }\ConstantTok{NA}\NormalTok{)}
\end{Highlighting}
\end{Shaded}

\begin{Shaded}
\begin{Highlighting}[]
\CommentTok{\# Sustituimos por todas las variables}

\NormalTok{worldSustainabiltyDataSet }\OtherTok{\textless{}{-}}\NormalTok{ worldSustainabiltyDataSet }\SpecialCharTok{\%\textgreater{}\%} \FunctionTok{replace}\NormalTok{(.}\SpecialCharTok{==}\StringTok{"N/A"}\NormalTok{, }\ConstantTok{NA}\NormalTok{)}

\NormalTok{worldSustainabiltyDataSet }\OtherTok{\textless{}{-}}\NormalTok{ worldSustainabiltyDataSet }\SpecialCharTok{\%\textgreater{}\%} \FunctionTok{replace}\NormalTok{(.}\SpecialCharTok{==}\StringTok{"{-}"}\NormalTok{, }\ConstantTok{NA}\NormalTok{)}

\NormalTok{worldSustainabiltyDataSet }\OtherTok{\textless{}{-}}\NormalTok{ worldSustainabiltyDataSet }\SpecialCharTok{\%\textgreater{}\%} \FunctionTok{replace}\NormalTok{(.}\SpecialCharTok{==}\StringTok{"NA"}\NormalTok{, }\ConstantTok{NA}\NormalTok{)}

\NormalTok{worldSustainabiltyDataSet }\OtherTok{\textless{}{-}}\NormalTok{ worldSustainabiltyDataSet }\SpecialCharTok{\%\textgreater{}\%} \FunctionTok{replace}\NormalTok{(.}\SpecialCharTok{==}\StringTok{""}\NormalTok{, }\ConstantTok{NA}\NormalTok{)}

\NormalTok{worldSustainabiltyDataSet }\OtherTok{\textless{}{-}}\NormalTok{ worldSustainabiltyDataSet }\SpecialCharTok{\%\textgreater{}\%} \FunctionTok{replace}\NormalTok{(.}\SpecialCharTok{==}\StringTok{"9999"}\NormalTok{, }\ConstantTok{NA}\NormalTok{)}

\NormalTok{worldSustainabiltyDataSet }\OtherTok{\textless{}{-}}\NormalTok{ worldSustainabiltyDataSet }\SpecialCharTok{\%\textgreater{}\%} \FunctionTok{replace}\NormalTok{(.}\SpecialCharTok{==}\StringTok{"Unknown"}\NormalTok{, }\ConstantTok{NA}\NormalTok{)}

\FunctionTok{names}\NormalTok{(worldSustainabiltyDataSet) }\OtherTok{\textless{}{-}}\NormalTok{ cleanNames}
\end{Highlighting}
\end{Shaded}

\subsection{2.2 Variables
cuantitativas}\label{variables-cuantitativas-1}

\subsubsection{2.2.1 Países y códigos de
países}\label{pauxedses-y-cuxf3digos-de-pauxedses}

\begin{Shaded}
\begin{Highlighting}[]
\CommentTok{\# definimos una función para hacer los cambios de formato}
\NormalTok{colDataFormatter}\OtherTok{\textless{}{-}} \ControlFlowTok{function}\NormalTok{(colToFix)\{}
\NormalTok{  colToFix }\OtherTok{\textless{}{-}} \FunctionTok{toupper}\NormalTok{(colToFix)}
\NormalTok{  colToFix }\OtherTok{\textless{}{-}} \FunctionTok{str\_replace\_all}\NormalTok{(colToFix, }\StringTok{" "}\NormalTok{, }\StringTok{"\_"}\NormalTok{)}
  \FunctionTok{return}\NormalTok{ (colToFix)}
\NormalTok{\}}
\end{Highlighting}
\end{Shaded}

\begin{Shaded}
\begin{Highlighting}[]
\NormalTok{countryCol }\OtherTok{\textless{}{-}}\NormalTok{ worldSustainabiltyDataSet}\SpecialCharTok{$}\NormalTok{Country}
\NormalTok{countryCodeCol }\OtherTok{\textless{}{-}}\NormalTok{ worldSustainabiltyDataSet}\SpecialCharTok{$}\NormalTok{CountryCode}

\CommentTok{\# Agregamos los cambios al dataset:}
\NormalTok{countryCol }\OtherTok{\textless{}{-}} \FunctionTok{colDataFormatter}\NormalTok{(countryCol)}
\NormalTok{countryCodeCol }\OtherTok{\textless{}{-}} \FunctionTok{colDataFormatter}\NormalTok{(countryCodeCol)}
\NormalTok{worldSustainabiltyDataSet}\SpecialCharTok{$}\NormalTok{Country }\OtherTok{\textless{}{-}}\NormalTok{ countryCol}
\NormalTok{worldSustainabiltyDataSet}\SpecialCharTok{$}\NormalTok{CountryCode }\OtherTok{\textless{}{-}}\NormalTok{ countryCodeCol}
\end{Highlighting}
\end{Shaded}

Cambios aplicados:

\begin{Shaded}
\begin{Highlighting}[]
\FunctionTok{head}\NormalTok{(}\FunctionTok{unique}\NormalTok{(countryCol),}\DecValTok{4}\NormalTok{)}
\end{Highlighting}
\end{Shaded}

\begin{verbatim}
## [1] "ARUBA"                "ANGOLA"               "ALBANIA"             
## [4] "UNITED_ARAB_EMIRATES"
\end{verbatim}

\begin{Shaded}
\begin{Highlighting}[]
\FunctionTok{head}\NormalTok{(}\FunctionTok{unique}\NormalTok{(countryCodeCol),}\DecValTok{4}\NormalTok{)}
\end{Highlighting}
\end{Shaded}

\begin{verbatim}
## [1] "ABW" "AGO" "ALB" "ARE"
\end{verbatim}

\subsubsection{2.2.2 Continente}\label{continente}

\begin{Shaded}
\begin{Highlighting}[]
\NormalTok{worldSustainabiltyDataSet}\SpecialCharTok{$}\NormalTok{Continent }\OtherTok{\textless{}{-}} \FunctionTok{colDataFormatter}\NormalTok{(worldSustainabiltyDataSet}\SpecialCharTok{$}\NormalTok{Continent)}
\end{Highlighting}
\end{Shaded}

\begin{Shaded}
\begin{Highlighting}[]
\FunctionTok{head}\NormalTok{(}\FunctionTok{unique}\NormalTok{(worldSustainabiltyDataSet}\SpecialCharTok{$}\NormalTok{Continent))}
\end{Highlighting}
\end{Shaded}

\begin{verbatim}
## [1] "NORTH_AMERICA" "AFRICA"        "EUROPE"        "ASIA"         
## [5] "SOUTH_AMERICA" "OCEANIA"
\end{verbatim}

\subsubsection{2.2.3 Régimen}\label{ruxe9gimen}

\begin{Shaded}
\begin{Highlighting}[]
\NormalTok{worldSustainabiltyDataSet}\SpecialCharTok{$}\NormalTok{Regime }\OtherTok{\textless{}{-}} \FunctionTok{colDataFormatter}\NormalTok{(worldSustainabiltyDataSet}\SpecialCharTok{$}\NormalTok{Regime)}
\end{Highlighting}
\end{Shaded}

\begin{Shaded}
\begin{Highlighting}[]
\FunctionTok{head}\NormalTok{(}\FunctionTok{unique}\NormalTok{(worldSustainabiltyDataSet}\SpecialCharTok{$}\NormalTok{Regime))}
\end{Highlighting}
\end{Shaded}

\begin{verbatim}
## [1] NA                    "CLOSED_AUTOCRACY"    "ELECTORAL_AUTOCRACY"
## [4] "ELECTORAL_DEMOCRACY" "LIBERAL_DEMOCRACY"
\end{verbatim}

\subsubsection{2.2.4 Región}\label{regiuxf3n}

\begin{Shaded}
\begin{Highlighting}[]
\NormalTok{worldSustainabiltyDataSet}\SpecialCharTok{$}\NormalTok{Region }\OtherTok{\textless{}{-}} \FunctionTok{colDataFormatter}\NormalTok{(worldSustainabiltyDataSet}\SpecialCharTok{$}\NormalTok{Region)}
\end{Highlighting}
\end{Shaded}

\begin{Shaded}
\begin{Highlighting}[]
\FunctionTok{head}\NormalTok{(}\FunctionTok{unique}\NormalTok{(worldSustainabiltyDataSet}\SpecialCharTok{$}\NormalTok{Region))}
\end{Highlighting}
\end{Shaded}

\begin{verbatim}
## [1] "LATIN_AMERICA_AND_CARIBBEAN"      "SUB-SAHARAN_AFRICA"              
## [3] "EUROPE_AND_NORTHERN_AMERICA"      "NORTHERN_AFRICA_AND_WESTERN_ASIA"
## [5] "OCEANIA"                          "CENTRAL_AND_SOUTHERN_ASIA"
\end{verbatim}

\subsubsection{2.2.5 Región}\label{regiuxf3n-1}

\begin{Shaded}
\begin{Highlighting}[]
\NormalTok{worldSustainabiltyDataSet}\SpecialCharTok{$}\NormalTok{Income }\OtherTok{\textless{}{-}} \FunctionTok{colDataFormatter}\NormalTok{(worldSustainabiltyDataSet}\SpecialCharTok{$}\NormalTok{Income)}
\end{Highlighting}
\end{Shaded}

\begin{Shaded}
\begin{Highlighting}[]
\FunctionTok{head}\NormalTok{(}\FunctionTok{unique}\NormalTok{(worldSustainabiltyDataSet}\SpecialCharTok{$}\NormalTok{Income))}
\end{Highlighting}
\end{Shaded}

\begin{verbatim}
## [1] "HIGH_INCOME"         "LOW_INCOME"          "LOWER-MIDDLE_INCOME"
## [4] "UPPER-MIDDLE_INCOME" NA
\end{verbatim}

\section{3 Valores Extremos}\label{valores-extremos}

\subsection{3.1 Desigualdad (GINI)}\label{desigualdad-gini}

\begin{Shaded}
\begin{Highlighting}[]
  \CommentTok{\# Renombramos las columnas}
  \CommentTok{\# Corresponde a la columna SI.POV.GINI}
\NormalTok{  worldSustainabiltyDataSet }\OtherTok{\textless{}{-}}\NormalTok{ worldSustainabiltyDataSet }\SpecialCharTok{\%\textgreater{}\%} \FunctionTok{rename}\NormalTok{( }\AttributeTok{GINI =} \StringTok{\textasciigrave{}}\AttributeTok{Gini.index..World.Bank.estimate....SI.POV.GINI}\StringTok{\textasciigrave{}}\NormalTok{)}
  
  \CommentTok{\# Corresponde al codigo GH.EM.IC.LUF}
\NormalTok{  worldSustainabiltyDataSet }\OtherTok{\textless{}{-}}\NormalTok{ worldSustainabiltyDataSet }\SpecialCharTok{\%\textgreater{}\%} \FunctionTok{rename}\NormalTok{( }\AttributeTok{GHE =} \StringTok{\textasciigrave{}}\AttributeTok{Annual.production.based.emissions.of.carbon.dioxide..CO2..measured.in.million.tonnes...GH.EM.IC.LUF}\StringTok{\textasciigrave{}}\NormalTok{)}
\end{Highlighting}
\end{Shaded}

\begin{Shaded}
\begin{Highlighting}[]
\CommentTok{\# Creamos una tabla con las 3 columnas a trabajar}
\CommentTok{\# Country GINI YEAR}

\NormalTok{tblCountryGiniYear }\OtherTok{\textless{}{-}}\NormalTok{ worldSustainabiltyDataSet }\SpecialCharTok{\%\textgreater{}\%}
  \FunctionTok{select}\NormalTok{(Country, GINI,Year)}
\CommentTok{\# Vistazo rápido de las columnas}
\FunctionTok{summary}\NormalTok{(tblCountryGiniYear}\SpecialCharTok{$}\NormalTok{Country)}
\end{Highlighting}
\end{Shaded}

\begin{verbatim}
##    Length     Class      Mode 
##      3287 character character
\end{verbatim}

\begin{Shaded}
\begin{Highlighting}[]
\FunctionTok{summary}\NormalTok{(tblCountryGiniYear}\SpecialCharTok{$}\NormalTok{GINI)}
\end{Highlighting}
\end{Shaded}

\begin{verbatim}
##    Min. 1st Qu.  Median    Mean 3rd Qu.    Max.    NA's 
##   23.70   30.90   35.60   37.51   42.80   64.80    1984
\end{verbatim}

\begin{Shaded}
\begin{Highlighting}[]
\FunctionTok{summary}\NormalTok{(tblCountryGiniYear}\SpecialCharTok{$}\NormalTok{Year)}
\end{Highlighting}
\end{Shaded}

\begin{verbatim}
##    Min. 1st Qu.  Median    Mean 3rd Qu.    Max. 
##    2000    2004    2009    2009    2014    2018
\end{verbatim}

Comenzaremos con la realización de una boxplot para visualizar los datos
de forma sencilla:

\begin{Shaded}
\begin{Highlighting}[]
\FunctionTok{boxplot}\NormalTok{(tblCountryGiniYear}\SpecialCharTok{$}\NormalTok{GINI, }\AttributeTok{main=}\StringTok{"GINI BOXPLOT"}\NormalTok{)}
\end{Highlighting}
\end{Shaded}

\pandocbounded{\includegraphics[keepaspectratio]{A1_Resolución_files/figure-latex/unnamed-chunk-21-1.pdf}}

\begin{Shaded}
\begin{Highlighting}[]
\FunctionTok{boxplot}\NormalTok{(tblCountryGiniYear}\SpecialCharTok{$}\NormalTok{Year, }\AttributeTok{main=}\StringTok{"Year BOXPLOT"}\NormalTok{)}
\end{Highlighting}
\end{Shaded}

\pandocbounded{\includegraphics[keepaspectratio]{A1_Resolución_files/figure-latex/unnamed-chunk-21-2.pdf}}

\begin{Shaded}
\begin{Highlighting}[]
\CommentTok{\# Comprobamos los stats para ver claramente los outliers}
\FunctionTok{boxplot.stats}\NormalTok{(tblCountryGiniYear}\SpecialCharTok{$}\NormalTok{GINI)}\SpecialCharTok{$}\NormalTok{out}
\end{Highlighting}
\end{Shaded}

\begin{verbatim}
## [1] 61.6 64.7 63.3 61.0 64.8 63.0 63.4 63.0
\end{verbatim}

\begin{Shaded}
\begin{Highlighting}[]
\CommentTok{\# Vamos a ver cuantos registr}
\FunctionTok{length}\NormalTok{(tblCountryGiniYear}\SpecialCharTok{$}\NormalTok{GINI)}
\end{Highlighting}
\end{Shaded}

\begin{verbatim}
## [1] 3287
\end{verbatim}

\begin{Shaded}
\begin{Highlighting}[]
\FunctionTok{length}\NormalTok{(}\FunctionTok{which}\NormalTok{(}\FunctionTok{is.na}\NormalTok{(tblCountryGiniYear}\SpecialCharTok{$}\NormalTok{GINI)))}
\end{Highlighting}
\end{Shaded}

\begin{verbatim}
## [1] 1984
\end{verbatim}

\begin{Shaded}
\begin{Highlighting}[]
\FunctionTok{length}\NormalTok{(}\FunctionTok{boxplot.stats}\NormalTok{(tblCountryGiniYear}\SpecialCharTok{$}\NormalTok{GINI)}\SpecialCharTok{$}\NormalTok{out)}
\end{Highlighting}
\end{Shaded}

\begin{verbatim}
## [1] 8
\end{verbatim}

Como podemos ver en el ``boxplot'' de GINI tenemos una gran cantidad de
valores atípicos, existiendo una asímetría de datos positiva, puesto que
tenemos bastantes valores outliers en GINI, esto conlleva cierta
dispersión ya que tenemos valores alejados de la media y mediana. Si nos
centramos en los valores del ``boxplot'' de Year podemos observar como
los mismos tienen cierta dispersión.

En cuanto a los valores outliers estos requieren una transformación
puesto que aunque son pocos pueden distorsionar el estudio de valores
esenciales como son la media/mediana, por lo que aplicamos
winsorización:

\begin{Shaded}
\begin{Highlighting}[]
\NormalTok{tblCountryGiniYear}\SpecialCharTok{$}\NormalTok{GINI }\OtherTok{\textless{}{-}}\NormalTok{ psych}\SpecialCharTok{::}\FunctionTok{winsor}\NormalTok{(}
  \AttributeTok{x =}\NormalTok{ tblCountryGiniYear}\SpecialCharTok{$}\NormalTok{GINI,}
  \AttributeTok{trim =} \FloatTok{0.01}
\NormalTok{)}

\FunctionTok{boxplot}\NormalTok{(tblCountryGiniYear}\SpecialCharTok{$}\NormalTok{GINI, }\AttributeTok{main=}\StringTok{"Wisorized GINI"}\NormalTok{)}
\end{Highlighting}
\end{Shaded}

\pandocbounded{\includegraphics[keepaspectratio]{A1_Resolución_files/figure-latex/unnamed-chunk-22-1.pdf}}

\subsection{3.2 Green House Emissions}\label{green-house-emissions}

\subsubsection{1ª Parte}\label{uxaa-parte}

Estudiaremos los valores outliers de forma similar al apartado 3.1:

\begin{Shaded}
\begin{Highlighting}[]
\FunctionTok{boxplot}\NormalTok{(worldSustainabiltyDataSet}\SpecialCharTok{$}\NormalTok{GHE, }\AttributeTok{main=}\StringTok{"GHE Boxplot"}\NormalTok{)}
\end{Highlighting}
\end{Shaded}

\pandocbounded{\includegraphics[keepaspectratio]{A1_Resolución_files/figure-latex/unnamed-chunk-23-1.pdf}}

\begin{Shaded}
\begin{Highlighting}[]
\FunctionTok{summary}\NormalTok{(worldSustainabiltyDataSet}\SpecialCharTok{$}\NormalTok{GHE)}
\end{Highlighting}
\end{Shaded}

\begin{verbatim}
##     Min.  1st Qu.   Median     Mean  3rd Qu.     Max.     NA's 
##    0.048    2.426   12.621  174.647   72.012 9956.569        2
\end{verbatim}

\begin{Shaded}
\begin{Highlighting}[]
\CommentTok{\# Comparamos cuantos outliers hay sobre la población.}
\FunctionTok{length}\NormalTok{(worldSustainabiltyDataSet}\SpecialCharTok{$}\NormalTok{GHE)}
\end{Highlighting}
\end{Shaded}

\begin{verbatim}
## [1] 3287
\end{verbatim}

\begin{Shaded}
\begin{Highlighting}[]
\FunctionTok{length}\NormalTok{(}\FunctionTok{boxplot.stats}\NormalTok{(worldSustainabiltyDataSet}\SpecialCharTok{$}\NormalTok{GHE)}\SpecialCharTok{$}\NormalTok{out)}
\end{Highlighting}
\end{Shaded}

\begin{verbatim}
## [1] 517
\end{verbatim}

Como podemos apreciar existe una gran cantidad de valores outliers
dentro de los datos, pero en este caso no tiene sentido su
transformación puesto que corresponde a aproximadamente el 16\% del
total de los mismos. En caso de realizar una transformación únicamente
conseguiriamos alterar el resultado del estudio, al eliminar tantos
resultados.

\subsubsection{2ª Parte}\label{uxaa-parte-1}

Seleccionamos en una lista compuesta por los países más contaminantes en
el año 2018, ordenandola de mayor a menor cantidad de emisiones.

\begin{Shaded}
\begin{Highlighting}[]
\CommentTok{\# Pillamos los valores de 2018}
\NormalTok{tblCountryGHE }\OtherTok{\textless{}{-}}\NormalTok{ worldSustainabiltyDataSet }\SpecialCharTok{\%\textgreater{}\%}
  \FunctionTok{filter}\NormalTok{(Year }\SpecialCharTok{==} \DecValTok{2018}\NormalTok{) }\SpecialCharTok{\%\textgreater{}\%}
  \FunctionTok{select}\NormalTok{(Country, GHE)}

\CommentTok{\# filtramos ahora por los valores outliers}

\NormalTok{tblCountryGHE }\OtherTok{\textless{}{-}}\NormalTok{ tblCountryGHE }\SpecialCharTok{\%\textgreater{}\%}
  \FunctionTok{filter}\NormalTok{(GHE }\SpecialCharTok{\textgreater{}=} \FunctionTok{min}\NormalTok{(}\FunctionTok{boxplot.stats}\NormalTok{(tblCountryGHE}\SpecialCharTok{$}\NormalTok{GHE)}\SpecialCharTok{$}\NormalTok{out) ) }\SpecialCharTok{\%\textgreater{}\%}
  \FunctionTok{arrange}\NormalTok{(}\FunctionTok{desc}\NormalTok{(GHE))}

\FunctionTok{head}\NormalTok{(tblCountryGHE)}
\end{Highlighting}
\end{Shaded}

\begin{verbatim}
##              Country      GHE
## 1              CHINA 9956.569
## 2      UNITED_STATES 5424.882
## 3              INDIA 2591.324
## 4 RUSSIAN_FEDERATION 1691.360
## 5              JAPAN 1135.688
## 6 IRAN,_ISLAMIC_REP.  755.402
\end{verbatim}

\section{4 Correlaciones}\label{correlaciones}

\begin{Shaded}
\begin{Highlighting}[]
\CommentTok{\# Codigos implicados: }
\CommentTok{\# SP.DYN.LE00.IN ==\textgreater{} LE [48]}
\CommentTok{\# SN\_ITK\_DEFC ==\textgreater{} DEFC [36]}
\CommentTok{\# SP\_ACS\_BSRVH2O ==\textgreater{} BSRVW [40]}
\CommentTok{\# SI\_POV\_DAY1 ==\textgreater{} POVL [37]}
\CommentTok{\# SE.PRM.UNER.ZS ==\textgreater{} COSCH [13]}
\CommentTok{\# NY.GDP.MKTP.CD ==\textgreater{} GDP [19]}

\CommentTok{\# Renombramos y extraemos para mayor facilidad}

\NormalTok{worldSustainabiltyDataSet }\OtherTok{\textless{}{-}}\NormalTok{ worldSustainabiltyDataSet }\SpecialCharTok{\%\textgreater{}\%} \FunctionTok{rename}\NormalTok{( }\AttributeTok{LE =} \StringTok{\textasciigrave{}}\AttributeTok{Life.expectancy.at.birth..total..years....SP.DYN.LE00.IN}\StringTok{\textasciigrave{}}\NormalTok{)}

\NormalTok{worldSustainabiltyDataSet }\OtherTok{\textless{}{-}}\NormalTok{ worldSustainabiltyDataSet }\SpecialCharTok{\%\textgreater{}\%} \FunctionTok{rename}\NormalTok{( }\AttributeTok{DEFC =} \StringTok{\textasciigrave{}}\AttributeTok{Prevalence.of.undernourishment.......SN\_ITK\_DEFC}\StringTok{\textasciigrave{}}\NormalTok{)}

\NormalTok{worldSustainabiltyDataSet }\OtherTok{\textless{}{-}}\NormalTok{ worldSustainabiltyDataSet }\SpecialCharTok{\%\textgreater{}\%} \FunctionTok{rename}\NormalTok{( }\AttributeTok{BSRVW =} \StringTok{\textasciigrave{}}\AttributeTok{Proportion.of.population.using.basic.drinking.water.services.......SP\_ACS\_BSRVH2O}\StringTok{\textasciigrave{}}\NormalTok{)}

\NormalTok{worldSustainabiltyDataSet }\OtherTok{\textless{}{-}}\NormalTok{ worldSustainabiltyDataSet }\SpecialCharTok{\%\textgreater{}\%} \FunctionTok{rename}\NormalTok{( }\AttributeTok{POVL =} \StringTok{\textasciigrave{}}\AttributeTok{Proportion.of.population.below.international.poverty.line.......SI\_POV\_DAY1}\StringTok{\textasciigrave{}}\NormalTok{)}

\NormalTok{worldSustainabiltyDataSet }\OtherTok{\textless{}{-}}\NormalTok{ worldSustainabiltyDataSet }\SpecialCharTok{\%\textgreater{}\%} \FunctionTok{rename}\NormalTok{( }\AttributeTok{COSCH =} \StringTok{\textasciigrave{}}\AttributeTok{Children.out.of.school....of.primary.school.age....SE.PRM.UNER.ZS}\StringTok{\textasciigrave{}}\NormalTok{)}

\NormalTok{worldSustainabiltyDataSet }\OtherTok{\textless{}{-}}\NormalTok{ worldSustainabiltyDataSet }\SpecialCharTok{\%\textgreater{}\%} \FunctionTok{rename}\NormalTok{( }\AttributeTok{GDP =} \StringTok{\textasciigrave{}}\AttributeTok{GDP..current.US.....NY.GDP.MKTP.CD}\StringTok{\textasciigrave{}}\NormalTok{)}

\NormalTok{colsToCorr }\OtherTok{\textless{}{-}} \FunctionTok{c}\NormalTok{(}\StringTok{"LE"}\NormalTok{, }\StringTok{"DEFC"}\NormalTok{, }\StringTok{"BSRVW"}\NormalTok{, }\StringTok{"POVL"}\NormalTok{, }\StringTok{"COSCH"}\NormalTok{, }\StringTok{"GDP"}\NormalTok{ )}

\NormalTok{dataToCorr }\OtherTok{\textless{}{-}}\NormalTok{ worldSustainabiltyDataSet }\SpecialCharTok{\%\textgreater{}\%} \FunctionTok{select}\NormalTok{(}\FunctionTok{all\_of}\NormalTok{(colsToCorr))}

\FunctionTok{head}\NormalTok{(dataToCorr)}
\end{Highlighting}
\end{Shaded}

\begin{verbatim}
##       LE DEFC BSRVW POVL   COSCH        GDP
## 1     NA   NA    NA   NA 1.60268 1873452514
## 2 73.853   NA    NA   NA 0.32258 1920111732
## 3 73.937   NA    NA   NA 1.81634 1941340782
## 4 74.038   NA    NA   NA 3.32156 2021229050
## 5 74.156   NA    NA   NA 2.17652 2228491620
## 6 74.287   NA    NA   NA 1.64077 2330726257
\end{verbatim}

\begin{Shaded}
\begin{Highlighting}[]
\NormalTok{corrMatrix }\OtherTok{\textless{}{-}} \FunctionTok{cor}\NormalTok{(dataToCorr, }
                  \AttributeTok{use =} \StringTok{"pairwise.complete.obs"}\NormalTok{) }\CommentTok{\#Ya que tiene elementos NA}
\NormalTok{corrMatrix }\OtherTok{\textless{}{-}} \FunctionTok{round}\NormalTok{(corrMatrix,}\DecValTok{2}\NormalTok{) }\CommentTok{\# Para tratar con 2 decimales}

\NormalTok{corrMatrix}
\end{Highlighting}
\end{Shaded}

\begin{verbatim}
##          LE  DEFC BSRVW  POVL COSCH   GDP
## LE     1.00 -0.57  0.77 -0.77 -0.66  0.20
## DEFC  -0.57  1.00 -0.67  0.68  0.42 -0.17
## BSRVW  0.77 -0.67  1.00 -0.80 -0.58  0.14
## POVL  -0.77  0.68 -0.80  1.00  0.69 -0.12
## COSCH -0.66  0.42 -0.58  0.69  1.00 -0.12
## GDP    0.20 -0.17  0.14 -0.12 -0.12  1.00
\end{verbatim}

\begin{Shaded}
\begin{Highlighting}[]
\CommentTok{\# Usamos la libreria corrplot para mostrar de forma simplificada la matriz}
\FunctionTok{corrplot}\NormalTok{(corrMatrix, }
         \AttributeTok{type =} \StringTok{"upper"}\NormalTok{, }
         \AttributeTok{order =} \StringTok{"hclust"}\NormalTok{,  }
         \AttributeTok{tl.col =} \StringTok{"black"}\NormalTok{, }
         \AttributeTok{tl.srt =} \DecValTok{45}
\NormalTok{         )}
\end{Highlighting}
\end{Shaded}

\pandocbounded{\includegraphics[keepaspectratio]{A1_Resolución_files/figure-latex/unnamed-chunk-25-1.pdf}}

Por los resultados obtenidos existe cierta correlación, una correlación
relativamente elevada puesto que la mayor correlación que existe es una
correlación negativa entre las variables POVL (población viviendo bajo
el umbral de la pobreza) y LE (Expectativa de Vida) y POVL con BSRVW
(Población que usa fuentes básicas de agua), siendo la primera de -0.77
(POVL \textless-\textgreater{} LE) y -0.8 (POVL \textless-\textgreater{}
BSRW).

Es decir, la esperanza de vida está fuertemente correlacionada con los
indicadores sociales y de servicios básicos (Pobreza, Educación y Acceso
al Agua) más que con el indicador económico crudo (PIB). Esto sugiere
que la inversión en bienestar social y servicios esenciales es más
efectiva para mejorar la salud de la población que el crecimiento
económico general por sí solo.

\section{5 Imputación}\label{imputaciuxf3n}

Con respecto a este apartado usaremos las medias de los vecinos más
cercanos usando las variables con una mayor correlación, es decir,
mediante las conclusiones obtenidas en el apartado anterior tenemos que
las variables a utilizar serán: BSRVW, POVL y COSCH puesto que son
aquellas con una correlación lo suficientemente alta.

\begin{Shaded}
\begin{Highlighting}[]
\CommentTok{\# Comprobamos que columnas estan vacías para aplicarle la imputación}
\NormalTok{tblNAYear }\OtherTok{\textless{}{-}}\NormalTok{ worldSustainabiltyDataSet }\SpecialCharTok{\%\textgreater{}\%}
  \FunctionTok{group\_by}\NormalTok{(Year) }\SpecialCharTok{\%\textgreater{}\%}
  \FunctionTok{filter}\NormalTok{(}\FunctionTok{all}\NormalTok{(}\FunctionTok{is.na}\NormalTok{(LE)) }\SpecialCharTok{==} \ConstantTok{TRUE}\NormalTok{) }\SpecialCharTok{\%\textgreater{}\%}
  \FunctionTok{select}\NormalTok{(Year) }\SpecialCharTok{\%\textgreater{}\%}
  \FunctionTok{ungroup}\NormalTok{()}

\FunctionTok{unique}\NormalTok{(tblNAYear}\SpecialCharTok{$}\NormalTok{Year)}
\end{Highlighting}
\end{Shaded}

\begin{verbatim}
## [1] 2000
\end{verbatim}

\begin{Shaded}
\begin{Highlighting}[]
\CommentTok{\# Creamos una tabla con los LE del año 2001 ya que no puede tener NA a la hora de hacer kNN}
\NormalTok{datoskNNImputados }\OtherTok{\textless{}{-}}\NormalTok{ worldSustainabiltyDataSet }\SpecialCharTok{\%\textgreater{}\%} \FunctionTok{filter}\NormalTok{(Year }\SpecialCharTok{==} \DecValTok{2001}\NormalTok{) }\SpecialCharTok{\%\textgreater{}\%} 
  \FunctionTok{select}\NormalTok{(LE, BSRVW, POVL, COSCH )}

\CommentTok{\# 2. Aplicar kNN. Imputamos LE (variable)}
\NormalTok{datoskNNClean }\OtherTok{\textless{}{-}} \FunctionTok{kNN}\NormalTok{(}
  \AttributeTok{data =}\NormalTok{ datoskNNImputados,}
  \AttributeTok{variable =} \FunctionTok{c}\NormalTok{(}\StringTok{"LE"}\NormalTok{),}
  \AttributeTok{k =} \DecValTok{5}\NormalTok{,}
  \AttributeTok{dist\_var =} \FunctionTok{c}\NormalTok{(}\StringTok{"BSRVW"}\NormalTok{, }\StringTok{"POVL"}\NormalTok{, }\StringTok{"COSCH"}\NormalTok{)}
\NormalTok{)}
\end{Highlighting}
\end{Shaded}

\begin{verbatim}
##    BSRVW     POVL    COSCH    BSRVW     POVL    COSCH 
##  29.0000   0.0000   0.0000 100.0000  68.4000  69.4751
\end{verbatim}

\begin{Shaded}
\begin{Highlighting}[]
\CommentTok{\#Aqui tenemos los datos de KNN con datos ya imputados}
\FunctionTok{head}\NormalTok{(datoskNNClean}\SpecialCharTok{$}\NormalTok{LE)}
\end{Highlighting}
\end{Shaded}

\begin{verbatim}
## [1] 73.853 47.059 74.288 74.544 73.755 71.800
\end{verbatim}

\begin{Shaded}
\begin{Highlighting}[]
\CommentTok{\# Preparamos la sustitución agregando una columna }
\NormalTok{worldSustainabiltyDataSet}\SpecialCharTok{$}\NormalTok{LE2000 }\OtherTok{\textless{}{-}}\NormalTok{ datoskNNClean}\SpecialCharTok{$}\NormalTok{LE}

\CommentTok{\# Hacemos un mutate para sustituir los valores del año 2000 por los valores guardados en LE2000}
\NormalTok{worldSustainabiltyDataSet }\OtherTok{\textless{}{-}}\NormalTok{ worldSustainabiltyDataSet }\SpecialCharTok{\%\textgreater{}\%}
  \FunctionTok{mutate}\NormalTok{(}
    \AttributeTok{LE =} \FunctionTok{case\_when}\NormalTok{(}
      \CommentTok{\# Condición: Si el año es 2000, usa el valor imputado}
\NormalTok{      Year }\SpecialCharTok{==} \DecValTok{2000} \SpecialCharTok{\textasciitilde{}}\NormalTok{ LE2000,}
      \CommentTok{\# Si el año no es 2000, mantén el valor original de la columna LE}
      \ConstantTok{TRUE} \SpecialCharTok{\textasciitilde{}}\NormalTok{ LE}
\NormalTok{    )}
\NormalTok{  ) }\SpecialCharTok{\%\textgreater{}\%}
  \FunctionTok{select}\NormalTok{(}\SpecialCharTok{{-}}\NormalTok{LE2000) }\CommentTok{\# Eliminar la columna auxiliar}
\end{Highlighting}
\end{Shaded}

\section{6 Tabla resumen}\label{tabla-resumen}

\begin{Shaded}
\begin{Highlighting}[]
\CommentTok{\# Variables relacionadas:}
\CommentTok{\# SI\_POV\_DAY1 {-}{-}\textgreater{} POVL}
\CommentTok{\# GINI}
\CommentTok{\# LE }

\CommentTok{\# Se pide calcular las medidas de tendencia central y dispersión (robustas y no robustas)}



\CommentTok{\# Datos: Agrupamos los datos por regiones, mostramos solo los datos más recientes.}

\CommentTok{\# Comenzamos agrupando los distintos datos:}

\NormalTok{recentData }\OtherTok{\textless{}{-}}\NormalTok{ worldSustainabiltyDataSet }\SpecialCharTok{\%\textgreater{}\%} 
  \FunctionTok{group\_by}\NormalTok{(Region) }\SpecialCharTok{\%\textgreater{}\%}
  \FunctionTok{filter}\NormalTok{(Year }\SpecialCharTok{==} \FunctionTok{max}\NormalTok{(Year, }\AttributeTok{na.rm =} \ConstantTok{TRUE}\NormalTok{)) }\SpecialCharTok{\%\textgreater{}\%}
  \FunctionTok{select}\NormalTok{(POVL, GINI, LE, Region)}


\CommentTok{\# Realizamos los cálculos al mismo tiempo y después separamos en tablas}
\NormalTok{recentData }\OtherTok{\textless{}{-}}\NormalTok{ recentData }\SpecialCharTok{\%\textgreater{}\%}
  \FunctionTok{summarise}\NormalTok{(}
    \CommentTok{\# Aplicar las cuatro funciones a las variables definidas}
    \FunctionTok{across}\NormalTok{(}
      \AttributeTok{.cols =} \FunctionTok{all\_of}\NormalTok{(}\FunctionTok{c}\NormalTok{(}\StringTok{\textquotesingle{}LE\textquotesingle{}}\NormalTok{, }\StringTok{\textquotesingle{}GINI\textquotesingle{}}\NormalTok{, }\StringTok{\textquotesingle{}POVL\textquotesingle{}}\NormalTok{)), }
      \AttributeTok{.fns =} \FunctionTok{list}\NormalTok{(}
        \CommentTok{\# TB1: Tendencia Central}
        \AttributeTok{Media =} \SpecialCharTok{\textasciitilde{}} \FunctionTok{mean}\NormalTok{(.x, }\AttributeTok{na.rm =} \ConstantTok{TRUE}\NormalTok{), }\CommentTok{\# No Robusta}
        \AttributeTok{Mediana =} \SpecialCharTok{\textasciitilde{}} \FunctionTok{median}\NormalTok{(.x, }\AttributeTok{na.rm =} \ConstantTok{TRUE}\NormalTok{), }\CommentTok{\# Robusta}
        
        \CommentTok{\# Dispersión}
        \AttributeTok{Desviacion\_Estandar =} \SpecialCharTok{\textasciitilde{}} \FunctionTok{sd}\NormalTok{(.x, }\AttributeTok{na.rm =} \ConstantTok{TRUE}\NormalTok{), }\CommentTok{\# No Robusta}
        \AttributeTok{Desviacion\_Absoluta\_Mediana =} \SpecialCharTok{\textasciitilde{}} \FunctionTok{mad}\NormalTok{(.x, }\AttributeTok{na.rm =} \ConstantTok{TRUE}\NormalTok{) }\CommentTok{\# Robusta}
\NormalTok{      ),}
      \AttributeTok{.names =} \StringTok{"\{.col\}\_\{.fn\}"} \CommentTok{\#  columnas dinámicas (\textquotesingle{}LE\_Media\textquotesingle{}, \textquotesingle{}GINI\_Mediana\textquotesingle{},... )}
\NormalTok{    ),}
    \AttributeTok{.groups =} \StringTok{\textquotesingle{}drop\textquotesingle{}} \CommentTok{\# No requerimos hacer agrupaciones tras realizar el cálculo.}
\NormalTok{  )}

\CommentTok{\# Tabla 1 : Medidas de tendencia Central (media, mediana)}

\NormalTok{centralTendencyTable }\OtherTok{\textless{}{-}}\NormalTok{ recentData }\SpecialCharTok{\%\textgreater{}\%} 
  \FunctionTok{select}\NormalTok{(}
\NormalTok{    Region,}
    \FunctionTok{ends\_with}\NormalTok{(}\StringTok{"Media"}\NormalTok{),}
    \FunctionTok{ends\_with}\NormalTok{(}\StringTok{"Mediana"}\NormalTok{)}
\NormalTok{  )}

\FunctionTok{print}\NormalTok{(centralTendencyTable)}
\end{Highlighting}
\end{Shaded}

\begin{verbatim}
## # A tibble: 8 x 10
##   Region        LE_Media GINI_Media POVL_Media LE_Mediana LE_Desviacion_Absolu~1
##   <chr>            <dbl>      <dbl>      <dbl>      <dbl>                  <dbl>
## 1 CENTRAL_AND_~     72.5       32.3      1.38        71.5                   1.98
## 2 EASTERN_AND_~     75.4       37.3      3.1         75.5                   8.73
## 3 EUROPE_AND_N~     79.1       31.0      0.448       80.9                   3.61
## 4 LATIN_AMERIC~     74.9       45.4      3.21        75.0                   2.68
## 5 NORTHERN_AFR~     76.0       34.3      1.18        76.5                   3.04
## 6 OCEANIA           74.3      NaN      NaN           71.8                   4.43
## 7 SUB-SAHARAN_~     62.6       38.6     33.1         63.0                   4.40
## 8 <NA>             NaN        NaN       38.5         NA                    NA   
## # i abbreviated name: 1: LE_Desviacion_Absoluta_Mediana
## # i 4 more variables: GINI_Mediana <dbl>,
## #   GINI_Desviacion_Absoluta_Mediana <dbl>, POVL_Mediana <dbl>,
## #   POVL_Desviacion_Absoluta_Mediana <dbl>
\end{verbatim}

\begin{Shaded}
\begin{Highlighting}[]
\CommentTok{\# Tabla 2 : Medias de dispersión (desviación estandar, desviación absoluta con respecto a la mediana)}

\NormalTok{dispersionTable }\OtherTok{\textless{}{-}}\NormalTok{ recentData }\SpecialCharTok{\%\textgreater{}\%} 
  \FunctionTok{select}\NormalTok{(}
\NormalTok{    Region, }
    \FunctionTok{ends\_with}\NormalTok{(}\StringTok{"Desviacion\_Estandar"}\NormalTok{), }
    \FunctionTok{ends\_with}\NormalTok{(}\StringTok{"Desviacion\_Absoluta\_Mediana"}\NormalTok{)}
\NormalTok{  )}

\FunctionTok{print}\NormalTok{(dispersionTable)}
\end{Highlighting}
\end{Shaded}

\begin{verbatim}
## # A tibble: 8 x 7
##   Region    LE_Desviacion_Estandar GINI_Desviacion_Esta~1 POVL_Desviacion_Esta~2
##   <chr>                      <dbl>                  <dbl>                  <dbl>
## 1 CENTRAL_~                   3.31                   6.73                  2.03 
## 2 EASTERN_~                   6.16                   3.23                  3.64 
## 3 EUROPE_A~                   3.51                   4.61                  0.733
## 4 LATIN_AM~                   3.31                   4.35                  3.91 
## 5 NORTHERN~                   3.13                   5.78                  1.95 
## 6 OCEANIA                     6.44                  NA                    NA    
## 7 SUB-SAHA~                   5.03                   8.64                 22.2  
## 8 <NA>                       NA                     NA                    NA    
## # i abbreviated names: 1: GINI_Desviacion_Estandar, 2: POVL_Desviacion_Estandar
## # i 3 more variables: LE_Desviacion_Absoluta_Mediana <dbl>,
## #   GINI_Desviacion_Absoluta_Mediana <dbl>,
## #   POVL_Desviacion_Absoluta_Mediana <dbl>
\end{verbatim}

\end{document}
